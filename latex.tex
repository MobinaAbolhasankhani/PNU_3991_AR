

\noindent 

\noindent 

\noindent 

\noindent 

\noindent \textbf{}

\noindent \textbf{Regular Expression {\textbar}  277}

\noindent \includegraphics*[width=3.80in, height=1.07in, keepaspectratio=false, trim=0.48in 0.65in 0.56in 0.99in]{image1}\textbf{}

\noindent \textbf{}

\noindent \textbf{}

\noindent \textbf{\textit{Step III}:} Between (a+b)* and (ab+ba), there is a .(dot) sign, and so an extra state is added

\noindent \includegraphics*[width=3.38in, height=1.28in, keepaspectratio=false, trim=0.62in 0.40in 0.64in 0.93in]{image2}between q${}_{0}$ and q${}_{f}$${}_{.}$   

\noindent .\textbf{}

\noindent \textbf{}

\noindent \textbf{}

\noindent \textbf{\textit{Step IV}}: Between ab and ba, there is + sign, and so there will be parallel edges between q${}_{1}$

\noindent \includegraphics*[width=3.04in, height=1.12in, keepaspectratio=false, trim=0.54in 0.51in 0.60in 0.72in]{image3}      and q${}_{f}$${}_{*}$

\noindent \textbf{.}

\noindent \textbf{}

\noindent \includegraphics*[width=3.28in, height=1.57in, keepaspectratio=false, trim=0.62in 0.32in 0.61in 0.63in]{image4}\textbf{\textit{Step V:}} Between `a' and b, there is a + sign. So, between q0 and qf there is a parallel edge. 

\noindent \textbf{}

\noindent \textbf{}

\noindent \textbf{}

\noindent \textbf{\textit{Step VI}}: Between `a' and `b' and between `b' and `a', there are .(dots). So, two extra states are added between q${}_{1}$ and q${}_{f}$ . An extra state is added between q${}_{0}$ and q${}_{1}$ for a(a + b)*.\textbf{}

\noindent \includegraphics*[width=3.31in, height=1.89in, keepaspectratio=false, trim=0.53in 0.21in 0.53in 0.36in]{image5}\textbf{}

\noindent \textbf{}

\noindent \textbf{}

\noindent \textbf{}

\noindent \textbf{278\textit{{\textbar}} \textit{Introduction to Automata Theory, Formal Languages and Computation }}

\noindent \includegraphics*[width=3.40in, height=2.30in, keepaspectratio=false, trim=2.60in 1.09in 1.97in 1.09in]{image6}\textbf{\textit{step VII}}: The * between q${}_{4}$  and q${}_{1}$ is removed by adding an extra state with label a, b, and the $\mathrm{\in }$  transition from q${}_{4\ \ }$to that state and from that state to q${}_{1}$\textbf{.}

\noindent 

\noindent 

\noindent \textbf{\textit{}}

\noindent \textbf{\textit{}}

\noindent \textbf{\textit{}}

\noindent \textbf{\textit{}}

\noindent \textbf{\textit{}}

\noindent \includegraphics*[width=3.36in, height=1.93in, keepaspectratio=false, trim=1.46in 0.62in 0.74in 0.57in]{image7}\textbf{\textit{Step VIII}}: Removing $\mathrm{\in }$, the automata become

\noindent 

\noindent 

\noindent 

\noindent 

\noindent 

\noindent 

\noindent 9. construct an FA equivalent to the RE, L = (00+11)*11(0+1)*0.

\noindent   \textbf{\textit{Solution}}:

\noindent \includegraphics*[width=3.38in, height=0.79in, keepaspectratio=false, trim=0.67in 0.51in 0.19in 1.09in]{image8}\textbf{\textit{Step I}}: Take a beginning state q${}_{0}$ and a final state q${}_{f\ }$ Between the beginning and fi nal state, place the RE. 

\noindent     

\noindent 

\noindent \includegraphics*[width=5.60in, height=0.69in, keepaspectratio=false, trim=0.41in 0.97in 0.26in 1.86in]{image9}\textbf{\textit{Step II}} : There are four (dots) in between (00+11)* and I , I and I , I and (0+1)* , and (0+1)* and 0. So , the four extra states are added in between q${}_{0\ }$and q${}_{f*}$

\noindent 

\noindent 

\noindent 

\noindent \textbf{\textit{Step III}}: The * between q${}_{0\ }$and q${}_{1\ }$ is removed by adding an extra state with label 00 and 11 as Loop and $\mathrm{\in }$ transition from q0 to that state and from that state to q1 The same is applied for the removal of * between q3 and q4.

\noindent . \textbf{Regular Expression {\textbar} 279}

\noindent \includegraphics*[width=5.99in, height=1.52in, keepaspectratio=false, trim=0.42in 1.42in 0.42in 0.90in]{image10}

\noindent 

\noindent 

\noindent 

\noindent 

\noindent \includegraphics*[width=5.49in, height=1.56in, keepaspectratio=false, trim=0.56in 1.22in 0.21in 0.74in]{image11}\textbf{\textit{Step IV  }}Removing the + sign between 00 and 11, parallel edges are added and for two .(dots) signs (between 0, 0 and 1,1), two extra states are added.

\noindent 

\noindent 

\noindent 

\noindent 

\noindent 

\noindent \includegraphics*[width=4.68in, height=2.56in, keepaspectratio=false, trim=1.19in 0.41in 0.39in 0.55in]{image12}\textbf{\textit{Step V}}: Use the $\mathrm{\in }$ removal technique to find the corresponding DFA.

\noindent 

\noindent 

\noindent 

\noindent 

\noindent 

\noindent 

\noindent 

\noindent 

\noindent 10. Construct an FA for the RE 10+(0+11)0*1.

\noindent \includegraphics*[width=5.00in, height=1.80in, keepaspectratio=false, trim=1.41in 1.11in 0.84in 1.17in]{image13} \textbf{\textit{Solution:}}

\noindent \textbf{\textit{}}

\noindent \textbf{\textit{}}

\noindent \textbf{\textit{}}

\noindent \textbf{\textit{}}

\noindent \textbf{\textit{}}

\noindent \textbf{\textit{}}

\noindent \includegraphics*[width=3.74in, height=1.55in, keepaspectratio=false, trim=0.41in 0.36in 0.22in 0.58in]{image14}\textbf{280 {\textbar} Introduction to Automata Theory, Formal Languages and Computation}

\noindent \textbf{}

\noindent \includegraphics*[width=2.19in, height=0.36in, keepaspectratio=false, trim=1.06in 0.62in 0.97in 1.38in]{image15}\includegraphics*[width=2.46in, height=2.01in, keepaspectratio=false, trim=2.40in 0.55in 0.54in 0.48in]{image16}\includegraphics*[width=3.12in, height=1.86in, keepaspectratio=false, trim=0.46in 0.26in 0.92in 0.42in]{image17}\textbf{\textit{}}

\noindent 

\noindent 

\noindent 

\noindent 

\noindent 

\noindent 

\noindent 

\noindent 

\noindent 

\noindent 

\noindent  

\noindent 11. Convert an RE (0+1)*\eqref{GrindEQ__10_}+\eqref{GrindEQ__00_}*\eqref{GrindEQ__11_}* to an NFA with $\mathrm{\in }$ move.

\noindent \includegraphics*[width=4.72in, height=2.31in, keepaspectratio=false, trim=0.55in 0.42in 0.86in 0.55in]{image18}[Gujrat Technological University 2010]

\noindent \includegraphics*[width=1.31in, height=0.42in, keepaspectratio=false, trim=1.89in 0.51in 1.07in 1.46in]{image19}\includegraphics*[width=5.46in, height=2.01in, keepaspectratio=false, trim=0.87in 0.83in 0.50in 1.00in]{image20}

